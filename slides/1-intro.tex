\documentclass[aspectratio=169]{beamer}

\input{macros/macros}
%-------------- template --------------------------------------------------
\input{macros/beamerconf}
%----------------------------------------------------------------------------

%------ using pstricks (rnode etc) ------------------------------------------
% \usepackage{pstricks,pst-node,pst-text,pst-3d}

% % context
% \AtBeginSection[]
% {
%     \begin{frame}
%         \frametitle{Table of Contents}
%         \tableofcontents[currentsection]
%     \end{frame}
% }
% \author[Renato Neves]{Renato Neves}
% % logos of institutions
% \titlegraphic{
%   \begin{textblock*}{5cm}(6.7cm,7.45cm)
%      \includegraphics[scale=0.06]{images/uminho.png}\hspace*{.85cm}~%
%   \end{textblock*}
%   \begin{textblock*}{5cm}(9.4cm,7.45cm)
%     \includegraphics[scale=0.50]{images/haslab.pdf}
%   \end{textblock*}
% }
% % No date
% \date{}

\begin{document}

\setLecture{1}{System Verification: Introduction}
\frame[plain]{\titlepage}


% \title{Cyber-Physical Computation}

% \frame[plain]{\titlepage}

% \section{System Verification}


\section{What are Formal Methods?}


\begin{slide}{What are formal methods?}
  \centering

  \begin{minipage}{0.7\textwidth}
  \Large
  
  \begin{block}{}~\\[-1mm]
    \centering
    Formal methods are \structure{techniques} to
    \\
    model \alert{complex systems} using
    \\
    \alert{rigorous mathematical models}

  \end{block}
  \end{minipage}
  \\[7mm]

  \begin{minipage}{0.32\textwidth}
  \begin{alertblock}{Specification}~\\[-1mm]
    Define part of the system using a modelling language
  \end{alertblock}
  \end{minipage}
~~
  \begin{minipage}{0.3\textwidth}
  \begin{alertblock}{Verification}~\\[-1mm]
    Prove properties.\\Show correctness.\\Find bugs.
  \end{alertblock}
  \end{minipage}
~~
  \begin{minipage}{0.28\textwidth}
  \begin{exampleblock}{Implementation}~\\[-1mm]
    Generate correct code.\\~\\~
  \end{exampleblock}
  \end{minipage}

\end{slide}

\begin{frame}
  \huge\centering
  All formal models are \alert{wrong}
  \pause
  \\[5mm]
  ... but some of them are \structure{usefull}!
\end{frame}


% \begin{slide}{Many success cases}
%   \huge\centering

%   Cryptographic protocols
%   \\[5mm]
%   Distributed systems do not get stuck
%   \\[5mm]
%   Safety in railway systems
%   \\[5mm]
%   Type systems
%   \\[5mm]
%   \faded{\emph{[Tests: Validation (not Verification)]}}

% \end{slide}


\begin{frame}[t]\frametitle{Program verification vs. System verification}
  
  \frsplitt{
    \begin{exampleblock}{Program verification}
      \begin{itemize}
        \item software (code)
        \item + annotations (logic)
        \item + some user interaction
        \item = correctness proof
      \end{itemize}
    \end{exampleblock}
    \pause
  }{
    \begin{alertblock}{SYSTEM verification}
      \begin{itemize}
        \item system specification (model)
        \item + system requirements (logic)
        \item + some user interaction
        \item + fixing parameters/scenarios
        \item = correctness proof
      \end{itemize}
    \end{alertblock}
  }
  \bigskip
  In this course: we will focus on \alert{model-checking}


\end{frame}


\section{Contents of the module}

\begin{slide}{Syllabus}
  \frsplitt{
    \begin{itemize}
      \item Introduction to model-checking
      \item CCS: a simple language for concurrency
      \begin{itemize}
        \item Syntax
        \item Semantics
        \item Equivalence
        \item \alert{mCRL2}: modelling
      \end{itemize}
      \item Dynamic logic
      \begin{itemize}
        \item Syntax
        \item Semantics
        \item Relation with equivalence
        \item \alert{mCRL2}: verification
      \end{itemize}
    \end{itemize}
  }{
    \begin{itemize}
      \item  {Timed Automata}
      \begin{itemize}
        \item Syntax
        \item Semantics (composition, Zeno)
        \item Equivalence
        \item \structure{UPPAAL}: modelling
      \end{itemize}
      \item Temporal logics (LTL/CTL)
      \begin{itemize}
        \item Syntax
        \item Semantics
        \item \structure{UPPAAL}: verification
      \end{itemize}
      \item Probabilistic and stochastic systems
      \begin{itemize}
        \item Going probabilistic
        \item \structure{UPPAAL}: monte-carlo
      \end{itemize}
    \end{itemize}
  }
\end{slide}


\section{Logistics}

\begin{frame}{Useful information}
  Relevant class material and announcements will be posted on the website periodically

  \mycbox{\structure{\url{https://fm-dcc.github.io/sv2526}}}
  

  \alert{E-mail}
    \begin{itemize}
      \item \href{mailto:jose.proenca@fc.up.pt}{jose.proenca@fc.up.pt} 
    \end{itemize}
           

  \alert{Office hours} (please send an email the day before if you wish to meet):
  \begin{itemize}
    \item \emph{José Proença:} Thursday morning
  \end{itemize} 
\end{frame}

\begin{frame}{Assessment}
  Assessment will consist of
  \begin{itemize}
  \item \structure{70\%} -- an individual \alert{test} at the end (\emph{época normal});
  \item \structure{30\%} -- a \alert{group assignment} with 2 parts involving the use of the mCRL2 and the Uppaal model checkers; and %on modelling and analysis of real-timed systems via \texttt{Uppaal} (40\%)
  \item \structure{100\%} -- Final (optional) exam during the extra period (\emph{época de recurso}).
  \end{itemize}
\end{frame}


\section{What is model-checking?}


\begin{frame}[t]\frametitle{What is model-checking?}

  \mycbox{\Large Check \alert{Requirements} of a \structure{Model}}

  \medskip

  \mycbox{\Large using \textbf{\underline{Formal Methods}}}

\end{frame}

\begin{frame}[t]\frametitle{Example: coffee machine}
    
  \splittwo{0.49}{0.49}{
    \includegraphicsframed[width=\textwidth]{images/coffeemachine.jpg}
  }{\centering
    \mycbox{\Large $\alert{\mathcal{M}},\structure{s} \models \phi$}
       does the \alert{model}
    \\ $\alert{\mathcal{M}}$
    \\ in \structure{state}
    \\ $\structure{s}$
    \\ satisfies the \textbf{requirement}
    \\ $\phi$
  }
\end{frame}

\begin{frame}[t]\frametitle{Example: coffee machine - the MODEL}
  \splittwo{0.25}{0.73}{
    ~\\[15mm]
    \includegraphics[width=\textwidth]{images/coffeemachine.jpg}
  }{\centering
    \includegraphicsframed[width=\textwidth]{images/coffeemachine-model.pdf}  
    \\[8mm]
    {\Large\bf \alert{Actions} \hfill \structure{States} \hfill Propositions}
    \\[8mm]
    Just building the model is often a large contribution
  }
\end{frame}

\begin{frame}[t]\frametitle{Example: coffee machine - the REQUIREMENTS}
  \splittwo{0.25}{0.73}{
    ~\\[15mm]
    \includegraphics[width=\textwidth]{images/coffeemachine.jpg}
  }{\centering
    \includegraphicsframed[width=\textwidth]{images/coffeemachine-model.pdf}  
    \\[5mm]
    \begin{block}{$\mathcal{M}, S2 \models \mi{coin}$}
    means coin holds in state \emph{S2}     
    \end{block}

    \begin{block}{$\mathcal{M}, S1 \models \mi{[make~selection]~selection}$}
    means selection holds in every state reachable with \emph{``make selection''} from \emph{S1}
    \end{block}
  }
\end{frame}

\begin{frame}[t]\frametitle{Actions vs. States}
  % \splittwo{0.4}{0.6}{
  % \begin{equation*}
  %   \xymatrix{
  %   & q_1  \ar[ld]_{a}  \ar[rd]^{a}\\
  %   q_2 \ar[rr]^{c}  && q_3 \ar@(ur,dr)[]^{c}
  %   }
  % \end{equation*}
  % }{
  %   \begin{itemize}
  %     \item  desired/forbidden sequences of actions
  %     \item Process algebra to generate models
  %     \item $\mathcal{M},2 \models [a]\mi{false}$
  %   \end{itemize}
  % }
  \begin{block}{Focus on \alert{events}}
  ~\\[-1mm]
  \splittwo{0.2}{0.6}{
    \begin{tikzpicture}[aut]
      \node (q2) {$q_2$};
      \node[xshift=30mm] (q3) {$q_3$};
      \node[xshift=15mm,yshift=10mm,initial above] (q1){$q_1$};
      \path[->]
        (q1) edge  node[lbl,swap]{\alert{a}}  (q2) edge node[lbl]{\alert{a}} (q3)
        (q2) edge node[lbl]{\alert{c}} (q3)
        (q3) edge[loop right] node[lbl]{\alert{c}} ();
    \end{tikzpicture}
  }{
    \begin{itemize}
      \item  desired/forbidden sequences of actions
      \item Process algebra to generate models
      \item $\mathcal{M},q_2 \models [\alert{a}]\mi{false}$
    \end{itemize}
  }
  \end{block}

  \begin{block}{Focus on \structure{states}}
  ~\\[-1mm]
  \splittwo{0.2}{0.6}{
    \begin{tikzpicture}[aut]
      \node (q2) {$q_2^{\structure{q}}$};
      \node[xshift=30mm] (q3) {$q_3^{\structure{p,q}}$};
      \node[xshift=15mm,yshift=10mm,initial above] (q1){$q_1^{\structure{p}}$};
      \path[->]
        (q1) edge   (q2) edge (q3)
        (q2) edge (q3)
        (q3) edge[loop right] ();
    \end{tikzpicture}
  }{
    \begin{itemize}
      \item reachable/forbidden states
      \item Language/Diagram to generate models
      \item $\mathcal{M},q_1\models \structure{p}$ , $\mathcal{M},q_1\models F~G~\structure{p}$
    \end{itemize}
  }
  \end{block}    
\end{frame}

\begin{frame}[t]\frametitle{Note 1/2: Models and logics are tightly connected}
    \mycbox{\Large $\alert{\mathcal{M}},q_2 \models \structure{[a]\mi{false}}$}

    \bigskip
    \begin{itemize}
      \item \alert{Models} that safisty exactly the same \structure{requirements}:
      \\~~\textbf{equivalence} (e.g. bisimulation, trace equivalence)
      \item \alert{Models} that satisfy a subset of \structure{requirements}:
      \\~~\textbf{inclusion} (e.g. simulation, trace inclusion)
      \item A \alert{model} should only capture the necessary to show its \structure{requirements}.
    \end{itemize}
\end{frame}


\begin{frame}[t]\frametitle{Note 2/2: Chose your Model and Logic wisely}
  \mycbox{\Large $\alert{\mathcal{M}},q_2 \models \structure{[a]\mi{false}}$}

  \bigskip
  \begin{itemize}
    \item \textbf{Real-time:} how long it takes between actions
    \item \textbf{Differential dynamic:} state evolves using differential equations
    \item \textbf{Beliefs:} who knows what
    \item \textbf{Deontic:} obrigatory and permitted actions
    \item \textbf{Fuzzy:} other values instead of truth values
    \item \textbf{Probabilistic:} the odds of something occuring
    \item \emph{Many tools:} \alert{mCRL2}, \alert{UPPAAL}, Spin, NuSMV (NuXMV), TLA+, Maude, Storm, CPN (petri nets)
  \end{itemize}
\end{frame}

\end{document}
